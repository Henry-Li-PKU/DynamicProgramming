\documentclass[11pt]{extarticle}
\usepackage{fullpage,amsmath,amsfonts,microtype,nicefrac,amssymb, amsthm}
\usepackage[left=1in, bottom=1in, top=1in, right = 1in]{geometry}
\usepackage{textcomp}
\usepackage{mathpazo}
\usepackage{mathrsfs}
\usepackage[T1]{fontenc}
\usepackage[utf8]{inputenc}
\usepackage[english]{babel}
\usepackage{graphicx}

\usepackage{microtype}

\usepackage{bm}
\usepackage{dsfont}
\usepackage{enumerate}
\usepackage{ragged2e}

\setlength{\parindent}{24pt}
\setlength{\jot}{8pt}


\usepackage[shortlabels]{enumitem}


%% FOOTNOTES
\usepackage[bottom]{footmisc}
\usepackage{footnotebackref}


%% FIGURE ENVIRONMENT
%\graphicspath{{}}
\usepackage[margin=15pt, font=small, labelfont={bf}, labelsep=period]{caption}
\usepackage{subcaption}
\captionsetup[figure]{name={Figure}, position=above}
\usepackage{float}
\usepackage{epstopdf}


%% NEW COMMANDS
\renewcommand{\baselinestretch}{1.25} 
\renewcommand{\qedsymbol}{$\blacksquare$}
\newcommand{\R}{\mathbb{R}}
\newcommand{\indep}{\mathrel{\text{\scalebox{1.07}{$\perp\mkern-10mu\perp$}}}}
\renewcommand{\b}{\begin}
\newcommand{\e}{\end}

%% NEWTHEOREM
\theoremstyle{plain}
\newtheorem{thm}{Theorem}
\newtheorem{lem}[thm]{Lemma}
\newtheorem{prop}[thm]{Proposition}

\theoremstyle{definition}
\newtheorem{defn}[thm]{Definition}
\newtheorem{ex}[thm]{Example}
\newtheorem{remark}[thm]{Remark}
\newtheorem{cor}[thm]{Corollary}

%% LINKS and COLORS
\usepackage[dvipsnames]{xcolor}
\usepackage{hyperref}
\definecolor{myred}{RGB}{163, 32, 45}
\hypersetup{
	%backref=true,
	%pagebackref=true,
	colorlinks=true,
	urlcolor=myred,
	citecolor=myred, 
	linktoc=all,     
	linkcolor=myred,
}

%% TABLE OF CONTENTS
\addto\captionsenglish{
	\renewcommand{\contentsname}
	{}% This removes the heading over the table of contents.
}



%%%%%%%%%%%%%%%%%%%%%%%%%%%%%%%%%%%%%%%%%%%%%%%%%%%%%%%%%%%%
%%%%%%%%%%%%%%%%%%%%%%%%%%%%%%%%%%%%%%%%%%%%%%%%%%%%%%%%%%%%
%%%%%%%%%%%%%%%%%%%%%%            END PREAMBLE           %%%%%%%%%%%%%%%%%%%%%%%%
%%%%%%%%%%%%%%%%%%%%%%%%%%%%%%%%%%%%%%%%%%%%%%%%%%%%%%%%%%%%
%%%%%%%%%%%%%%%%%%%%%%%%%%%%%%%%%%%%%%%%%%%%%%%%%%%%%%%%%%%%

\title{202A: Dynamic Programming and Applications\\[5pt] {\Large \textbf{Homework \#3}}}

\author{Andreas Schaab}

\date{}


\begin{document}

\maketitle



%%%%%%%%%%%%%%%%%%%%%%%%%%%%%%%%%%%%%%%%%%%%%%%%%%%%%%%%%%%%
%%%%%%%%%%%%%%%%%%%%%%%%%%%%%%%%%%%%%%%%%%%%%%%%%%%%%%%%%%%%
\section*{Problem 1}

This problem collects several exercises on Brownian motion and stochastic calculus.

\begin{enumerate}[(a)]
\item Show that $\mathbb Cov(B_s, B_t) = \min\{s, t\}$ for two times $0 \leq s < t$. Use the following tricks: Use the covariance formula $\mathbb Cov(A, B) = \mathbb E (AB) - \mathbb E(A) \mathbb E(B)$. Use $B_t \sim \mathcal N(0, t)$ as well as $B_t - B_s \sim \mathcal N(0, t-s)$. And use $B_t = B_s + (B_t - B_s)$.

\item Geometric Brownian motion evolves as: $dX_t = \mu X_t dt + \sigma X_t dB_t$. Show that
\begin{equation*}
	X_t = X_0 e^{\mu t - \frac{\sigma^2}{2} t + \sigma B_t}
\end{equation*}
for a given initial value $X_0$.

\item For Geometric Brownian motion as defined above, show that $\mathbb E = X_0 e^{\mu t}$.

\item The Ornstein-Uhlenbeck (OU) process is like a continuous-time variant of the AR(1) process. It evolves as $dX_t = - \mu X_t dt + \sigma dB_t$ for drift parameter $\mu$, diffusion parameter $\sigma$, and some $X$. Show that it solves 
\begin{equation*}
	X_t = X_0 e^{- \mu t} + \sigma \int_0^t e^{-\mu(t - s)} dB_s.
\end{equation*}
\end{enumerate}


%%%%%%%%%%%%%%%%%%%%%%%%%%%%%%%%%%%%%%%%%%%%%%%%%%%%%%%%%%%%
%%%%%%%%%%%%%%%%%%%%%%%%%%%%%%%%%%%%%%%%%%%%%%%%%%%%%%%%%%%%
\vspace{10mm}
\section*{Problem 3: heuristic proof of Ito's lemma **}

\textbf{Credit:} derivation follows David Laibson's class notes. For a more formal treatment, see especially Oksendal. 

\vspace{5mm}
\noindent
Consider a pretty general Ito process
\begin{equation*}
	dx = a(x, t) dt + b(x, t) dB,
\end{equation*}
where $dB$ is standard Brownian motion. We call $a(\cdot)$ the drift and $b(\cdot)$ the diffusion coefficients. Suppose $x(t)$ describes the price of oil (or the stock market) over time. Let's denote by $V(x, t)$ the \textit{value} of an oil well (or a particular portfolio) at time $t$, given the value of oil. 

We would like to characterize how $V(x, t)$ evolves over time. If there was no uncertainty, we would just take a time derivative and have $\frac{d}{dt} V(x, t) = \partial_t V(x, t) + \partial_x V(x, t) \frac{dx}{dt}$ by the chain rule. With uncertainty, we will have one additional term. 

\begin{prop}
	Ito's lemma: We have 
	\begin{align*}
		dV &= \partial_t V dt + \partial_x V dx + \frac{1}{2} \partial_{xx} V b(x, t)^2 dt \\
		&= \bigg[ \partial_t V + \partial_x V a(x, t) + \frac{1}{2} \partial_{xx} V b(x, t)^2\bigg] dt + \partial_x V b(x, t) dB.
	\end{align*}
\end{prop}

\vspace{5mm}
\noindent
We will now work through a heuristic proof.
\begin{enumerate}
\item Consider the function $V(x, t)$ and take a Taylor expansion.

\item Use the fact that terms of order $(dt)^\frac{3}{2}$ become really small in continuous time and ``drop out''. Show that $dx dt$ drops out. What are we left with

\item Plug in for $dx$ and $(dx)^2$ and arrive at Ito's lemma. Use $(dB)^2 \sim dt$.

\end{enumerate}




%%%%%%%%%%%%%%%%%%%%%%%%%%%%%%%%%%%%%%%%%%%%%%%%%%%%%%%%%%%%
%%%%%%%%%%%%%%%%%%%%%%%%%%%%%%%%%%%%%%%%%%%%%%%%%%%%%%%%%%%%
\vspace{10mm}
\section*{Problem 4: more practice with Ito's lemma ** (Canonical)}

Take the setting from above where we try to model the value of an oil well. Suppose that 
\begin{equation*}
	V(x, t) = \log(x).
\end{equation*}

\begin{enumerate}
\item Derive $V'(x)$ and $V''(x)$. Why do we need them? Why are we suddenly working with ordinary derivative terms rather than partial derivatives? 

\item Suppose $dx = \alpha x dt + \sigma x dB$. Plug into Ito's Lemma.

\item Plug in for $V'(x)$ and $V''(x)$ and arrive at the solution for $dV$. 

\item Notice that $\mathbb E[ dV ] = \alpha - \frac{1}{2} \sigma^2$ because $\mathbb E (dB) = 0$. Interpret this. Why is the drift of $V$ less than $\alpha$ even thow the drift of the oil price is $\alpha$? (Hint: What is the important economic property of log preferences that we care about here?)

\end{enumerate}





%%%%%%%%%%%%%%%%%%%%%%%%%%%%%%%%%%%%%%%%%%%%%%%%%%%%%%%%%%%%
%%%%%%%%%%%%%%%%%%%%%%%%%%%%%%%%%%%%%%%%%%%%%%%%%%%%%%%%%%%%
\vspace{10mm}
\section*{Problem 5: let's derive the HJB **}

Let $u(x, c, t)$ denote the instantaneous utility flow of an agent. $x$ is a state variable and $c$ is the control variable. For the state variable $x$, assume the same Ito process as above. In the continuous time limit with $\Delta t$ time steps, let $x' = x + \Delta x$ and $t' = t + \Delta t$.

\begin{enumerate}
\item Write down the discrete-time Bellman equation with $\Delta t$ time steps. Explain each term where a $\Delta t$ shows up. 

\item Multiply by $(1 + \rho \Delta t)$ and simplify terms to arrive at an expression $\rho V \Delta t = \ldots$. 

\item Divide by $\Delta t$. Then take the continuous time limit $\Delta t \to dt$ and notice that $(dt)^2 \approx 0$. 

\item Use Ito's lemma to rewrite $\mathbb E[dV]$ as 
\begin{equation*}
	\mathbb E[dV] = \bigg[ \partial_t V + \partial_x V a + \frac{1}{2} \partial_{xx} V b^2 \bigg] dt.
\end{equation*}
We have now derived the HJB equation with risk:
\begin{equation*}
		\rho V(x, t) = \max_c \bigg\{ u(x, c, t) + \partial_t V + a \partial_x V + \frac{b^2}{2} \partial_{xx} V \bigg\}
\end{equation*}

\end{enumerate}

%%%%%%%%%%%%%%%%%%%%%%%%%%%%%%%%%%%%%%%%%%%%%%%%%%%%%%%%%%%%
%%%%%%%%%%%%%%%%%%%%%%%%%%%%%%%%%%%%%%%%%%%%%%%%%%%%%%%%%%%%
\vspace{10mm}
\section*{Problem 1: XXX}

\begin{enumerate}[(a)]
\item 

\end{enumerate}


%%%%%%%%%%%%%%%%%%%%%%%%%%%%%%%%%%%%%%%%%%%%%%%%%%%%%%%%%%%%
%%%%%%%%%%%%%%%%%%%%%%%%%%%%%%%%%%%%%%%%%%%%%%%%%%%%%%%%%%%%
\vspace{10mm}
\section*{Problem 2: XXX}

XXX

\begin{enumerate}[(a)]
\item 

\end{enumerate}




\end{document}











